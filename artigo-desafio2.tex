\documentclass[conference]{IEEEtran}
\IEEEoverridecommandlockouts
%----------------------------------------------------------
%\graphicspath{images/}
%\DeclareGraphicsExtensions{.pdf,.jpeg,.png,.jpg}
\usepackage{amsmath,amssymb,amsfonts}
\usepackage{algorithmic}
\usepackage{graphicx}
\usepackage{textcomp}
\usepackage{array}
%\usepackage[caption=false,font=normalsize,labelfont=sf,textfon =sf]{subfig}
\usepackage{dblfloatfix}
\usepackage{url}
\usepackage{lipsum}
\usepackage{listings}
\usepackage{xcolor}
\def\BibTeX{{\rm B\kern-.05em{\sc i\kern-.025em b}\kern-.08em
    T\kern-.1667em\lower.7ex\hbox{E}\kern-.125emX}}
%----------------------------------------------------------
\lstset{
    escapeinside={/*@}{@*/},
    language=Python,	
    basicstyle=\fontsize{8.5}{12}\selectfont,
    numbers=left,
    numbersep=2pt,    
    xleftmargin=2pt,
    frame=tb,
    columns=fullflexible,
    showstringspaces=false,
    tabsize=4,
    keepspaces=true,
    showtabs=false,
    showspaces=false,
    morekeywords={inline,public,class,private,protected,struct},
    captionpos=b,
    lineskip=-0.4em,
    aboveskip=10pt,
    extendedchars=true,
    breaklines=true,
    prebreak = \raisebox{0ex}[0ex][0ex]{\ensuremath{\hookleftarrow}},
    keywordstyle=\color[rgb]{0,0,1},
    commentstyle=\color[rgb]{0.133,0.545,0.133},
    stringstyle=\color[rgb]{0.627,0.126,0.941},
}
%----------------------------------------------------------
\begin{document}

\title{Desafio 2: Sistema Embarcado com Sensor e Display\\
{\footnotesize \textsuperscript{*} Sistemas Embarcados: Prof. Marco Reis - marco.reis@ba.docente.senai.br}
}

\author{\IEEEauthorblockN{Leonardo Marques Trinchão}
\IEEEauthorblockA{\textit{Senai CIMATEC} \\
\textit{Engenharia Elétrica}\\
Salvador, Bahia, Brazil \\
leonardo.trinchao@aln.senaicimatec.edu.br}
}

\maketitle

\begin{abstract}

\end{abstract}
    
\begin{IEEEkeywords}
    component, formatting, style, styling, insert
\end{IEEEkeywords}
    
\section{Introdução}
\subsection{Arduino}

    Arduino é uma plataforma de prototipagem eletrônica, cujo principal objetivo é incorporar a ele 
funções através de componentes como sensores, LED's, módulos, e outros. A partir disso, torna-se 
possível criar sistemas embarcados versáteis para projetos em eletrônica, que consistem em criar desde 
pequenos robôs com funções limitadas, até protótipos baseados na utilização de Internet das Coisas (IoT),
automações residenciais, e outros.

    Para realizar as ações desejadas pelo usuário, é necessário programar o Arduino. Para isso, utiliza-se 
a linguagem de programação C/C++. Após ser programado, ele funciona sem a necessidade de um computador, por
exemplo, devido à um \emph{loop} que existe em seu próprio código. Assim, a única exigência para o
funcionamento do sistema é que haja uma fonte de alimentação.\\

\subsubsection{Arduino Pricipal}

    O Arduino Principal/Arduino Master recebe este nome devido à sua importância e relevância no projeto.
Ele recebe este nome por ser o responsável por controlar todo o sistema embarcado, de forma que realiza
cálculos matemáticos, operações e comandos envolvendo bibliotecas do próprio Arduino.

\subsubsection{Arduino Secundário}

    O Arduino Secundário/Arduino Slave é um sistema microprocessado que atua como suporte para o 
Arduino Master, ou seja, sua função é complementar suas necessidades, de acordo com o que é exigido.
Para que esse suporte seja realizado com êxito, é necessário comunicar os dois Arduinos, e para isso
utiliza-se a Comunicação Serial.

\subsection{Comunicação Serial}

    É possível que, em um sistema, dois Arduinos exerçam um papel complementar, onde um não funciona
perfeitamente sem o outro. Para sintonizar os dois Arduinos, é necessário criar um meio de Comunicação
entre eles, e com isso surge a Comunicação Serial. Ela funciona de forma que um Arduino Principal realiza
a principal parte do código, e pode existir um ou mais Arduinos Complementares, que exercem, como o próprio
nome diz, funções complementares.

    A partir disso, o Arduino Principal envia uma informação, seja ela um valor inteiro, uma letra ou uma
\emph{string}, e o(s) Arduino(s) Complementar(es) recebem essa informação e a utilizam para realizar suas
demais funções, como exibir essa informação em um \emph{LCD Display}. 

    Para que essa comunicação seja possível, é necessário realizar alguns procedimentos antes, como:
    \begin{itemize}
        \item Conectar as portas TX (transmissão) e RX (recepção) dos Arduinos
        \item Conectar as portas \emph{Ground} dos Arduinos
        \item Uso da biblioteca \emph{SoftwareSerial.h} nos dois Arduinos
        \item Uso do comando \emph{Serial.begin(9600)} dentro da \emph{void setup}, também nos dois
        \item Uso de condição envolvendo \emph{Serial.available()} no Arduino Complementar
        \item Receber o valor impresso no monitor serial do Arduino Pricipal utilizando 
        \emph{Serial.readString()}
        \item Atribuir esse valor à uma variável do tipo \emph{String}
        \item Utilizar o comando \emph{Serial.flush()} para esperar toda a "passagem" da informação
    \end{itemize}

\subsection{Sensor Ultra-Sônico}

    O Sensor Ultra-Sônico é um componente utilizado para captar a distância de um objeto, entre os pontos 
de mínimo e máximo do sensor. O ponto de distância mínima é de 

\subsection{LED RGB}



\subsection{LCD Display}



\section{Objetivos}

\section{Desenvolvimento}

\section{Resultados}
    O Arduino Principal (ou Arduino Master), possui extrema importância quando almeja-se construir 
um Sistema Embarcado. Ele é o cérebro do sistema microprocessado, e é o responsável por controlar
os comandos executados em todo o sistema. Neste projeto, o Arduino Principal age de forma que proporciona
o funcionamento pleno do sensor, atrelado à LED RGB, que emite a cor desejada, de acordo com a 
distância que um objeto está do Sensor. Além disso, ele distingue qual a região de proximidade

\section{Conclusões}

\section{Referências}

\end{document}